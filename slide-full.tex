% \documentclass
\documentclass[aspectratio=169]{ctexbeamer}
\usepackage{listings}
\usepackage{amsmath}
\usepackage{pgfplots}
\usepackage{tikz}
\usepackage{graphicx}
\usepackage{booktabs}
\usepackage{siunitx} % Required for alignment
\sisetup{
  round-mode          = places, % Rounds numbers
  round-precision     = 2, % to 2 places
}

\mode<presentation>
{
  \usetheme{villa}
}

% 中括号里的简写会在每页底部显示
\title[Villa Beamer Theme]{A Beamer Theme for Peking University Visual-Information Interlligent Learning Lab}
\subtitle{这里是副标题} % 可选
\author[Duanyll]{报告人:Duanyll}
\date[\today]{\today}

% 包含下面的指令,在每个 Subsetsion 前显示目录.
\AtBeginSubsection[]
{
  \begin{frame}<beamer>
    \tableofcontents[currentsection,currentsubsection]
  \end{frame}
}

% 包含下面的指令,在每个 Section 前显示目录.
\AtBeginSection[]
{
  \begin{frame}<beamer>
    \tableofcontents[currentsection]
  \end{frame}
}
% 如果不需要显示 Section 的目录,还要使用下面的指令覆盖 beamer 的默认设置
% \AtBeginSection[] {}

% 如果不需要右下角的按钮,使用下面的指令
% 在不能全屏放映 PDF 的环境中(比如在浏览器中),这个按钮会很有用
% \setbeamertemplate{navigation symbols}{}

\begin{document}

\begin{frame}[plain,noframenumbering]
  \titlepage
\end{frame}

\begin{frame}
  \tableofcontents
\end{frame}

\section{外部样式}

\begin{frame}{幻灯片的标题}
  \begin{itemize}
    \item 左上角数字为当前 Section 编号
    \item 左上角圈内为当前 Section 的标题
    \item 幻灯片标题显示在圈的右侧
    \item 不显示当前 Subsection 的标题
  \end{itemize}
\end{frame}

\begin{frame}{还可以设置副标题}{我是副标题}
  副标题显示在标题下方。

  注意标题长度不能超过页面宽度,否则会导致标题栏排版错乱。
\end{frame}

\section{内部样式}

\subsection{基本样式}

\begin{frame}{文字样式}
  \begin{itemize}
    \item 这是默认的字体。(Windows 为微软雅黑,Linux 为 Fandol 黑体)
    \item 这是 \textbf{粗体}。
    \item 这是 \texttt{等宽字体(仿宋)}。
    \item 这是 \underline{下划线}。
    \item 这是 \alert{高亮文本}。
  \end{itemize}
\end{frame}

\begin{frame}{列表样式}
  \begin{columns}
    \column{0.33\textwidth}
    Itemize 环境:
    \begin{itemize}
      \item 第一项
      \item 第二项
      \begin{itemize}
        \item 嵌套第一项
        \item 嵌套第二项
      \end{itemize}
      \item 第三项
    \end{itemize}

    \column{0.33\textwidth}
    Enumerate 环境:
    \begin{enumerate}
      \item 第一项
      \item 第二项
      \begin{enumerate}
        \item 嵌套第一项
        \item 嵌套第二项
      \end{enumerate}
      \item 第三项
    \end{enumerate}

    \column{0.33\textwidth}
    Description 环境:
    \begin{description}
      \item[第一项] 我是描述。
      \item[第二项] 也是描述。
      \item[第三项] 描述。
      \item[] 没有标签。
    \end{description}
  \end{columns}
\end{frame}

\begin{frame}{块环境}
  这一页展示了三种块环境的样式。
  \begin{columns}[T,onlytextwidth]
    \column{0.5\textwidth}
      \begin{block}{Default}
        Block content.
      \end{block}

      \begin{alertblock}{Alert}
        Block content.
      \end{alertblock}

      \begin{exampleblock}{Example}
        Block content.
      \end{exampleblock}

    \column{0.5\textwidth}

      \begin{block}{Default}
        Block content.
      \end{block}

      \begin{alertblock}{Alert}
        Block content.
      \end{alertblock}

      \begin{exampleblock}{Example}
        Block content.
      \end{exampleblock}

  \end{columns}
\end{frame}

\begin{frame}[fragile]
  \frametitle{代码环境}
  使用 \texttt{lstlisting} 插入一些 Python 代码。记得给这页加上 \texttt{fragile} 选项。

  \begin{lstlisting}[language=Python,basicstyle=\small]
  def quicksort(arr):
      if len(arr) <= 1:
          return arr
      pivot = arr[len(arr) // 2]
      left = [x for x in arr if x < pivot]
      middle = [x for x in arr if x == pivot]
      right = [x for x in arr if x > pivot]
      return quicksort(left) + middle + quicksort(right)
  \end{lstlisting}
\end{frame}

\begin{frame}{公式}
  \begin{block}{无编号公式}
    \begin{equation*}
        J(\theta) = \mathbb{E}_{\pi_\theta}[G_t] = \sum_{s\in\mathcal{S}} d^\pi (s)V^\pi(s)=\sum_{s\in\mathcal{S}} d^\pi(s)\sum_{a\in\mathcal{A}}\pi_\theta(a|s)Q^\pi(s,a)
    \end{equation*}
  \end{block}
  \begin{block}{多行多列公式\footnotemark[1]}
      % 使用 & 分隔
      \begin{align}
          Q_\mathrm{target}&=r+\gamma Q^\pi(s^\prime, \pi_\theta(s^\prime)+\epsilon)\\
          \epsilon&\sim\mathrm{clip}(\mathcal{N}(0, \sigma), -c, c)\nonumber
      \end{align}
  \end{block}
  \footnotetext[1]{如果公式中有文字出现,请用 $\backslash$mathrm\{\} 或者 $\backslash$text\{\} 包含,不然就会变成 $clip$,在公式里看起来比 $\mathrm{clip}$ 丑非常多。}
\end{frame}

\subsection{图表和分栏}

\begin{frame}{插入 PDF 图片}
  首选使用 PDF 格式的矢量图,以保证图片质量。
  \begin{figure}
    \centering
    \includegraphics[width=0.8\textwidth]{imgs/pkuvilla.pdf}
    \caption{这个 Logo 是从 PowerPoint 中导出的,不完全是矢量图。}
  \end{figure}
  可以使用 Inkscape 等工具将 SVG 格式的矢量图转换为 PDF 格式。
\end{frame}

\begin{frame}{TikZ 绘图和分栏}
  \begin{columns}[T]
    \column{0.4\textwidth}
    \begin{enumerate}
      \item 这一页使用了 \texttt{columns} 环境,给 \texttt{column} 环境加上 \texttt{T} 选项可以使两栏顶部对齐。
      \item 这是一个有序列表。
      \item 下面是用 \texttt{tikz} 绘制的函数图像。
    \end{enumerate}
    \column{0.6\textwidth}
    \begin{figure}
      \newcounter{density}
      \setcounter{density}{20}
      \resizebox{0.6\columnwidth}{!}{
        \begin{tikzpicture}
          \def\couleur{alerted text.fg}
          \path[coordinate] (0,0)  coordinate(A)
                      ++( 90:5cm) coordinate(B)
                      ++(0:5cm) coordinate(C)
                      ++(-90:5cm) coordinate(D);
          \draw[fill=\couleur!\thedensity] (A) -- (B) -- (C) --(D) -- cycle;
          \foreach \x in {1,...,40}{%
              \pgfmathsetcounter{density}{\thedensity+20}
              \setcounter{density}{\thedensity}
              \path[coordinate] coordinate(X) at (A){};
              \path[coordinate] (A) -- (B) coordinate[pos=.10](A)
                                  -- (C) coordinate[pos=.10](B)
                                  -- (D) coordinate[pos=.10](C)
                                  -- (X) coordinate[pos=.10](D);
              \draw[fill=\couleur!\thedensity] (A)--(B)--(C)-- (D) -- cycle;
          }
        \end{tikzpicture}
      }
      \caption{Rotated square from
      \href{http://www.texample.net/tikz/examples/rotated-polygons/}{texample.net}.}
    \end{figure}
  \end{columns}
\end{frame}

\begin{frame}{表格}
  \begin{columns}
    \column{0.3\textwidth}
    \begin{table}
      \centering
      \begin{tabular}{|c|c|c|}
        \hline
        A & B & C \\
        \hline
        1 & 2 & 3 \\
        4 & 5 & 6 \\
        7 & 8 & 9 \\
        \hline
      \end{tabular}
      \caption{这是一个简单的表格。}
    \end{table}
    \column{0.7\textwidth}
    \begin{table}
      \centering
      \begin{tabular}{l|S|r}
        \toprule % <-- Toprule here
        \textbf{Value 1} & \textbf{Value 2} & \textbf{Value 3}\\
        $\alpha$ & $\beta$ & $\gamma$ \\
        \midrule % <-- Midrule here
        1 & 1110.1 & a\\
        2 & 10.1 & b\\
        3 & 23.113231 & c\\
        \bottomrule % <-- Bottomrule here
      \end{tabular}
      \label{tab:table1}
      \caption{这是一个三线表,设置了更复杂的对齐规则。}
    \end{table}
  \end{columns}
\end{frame}

\section{动画效果}

\subsection{Overlay Specifications}

\begin{frame}{Overlay Specifications}
  使用 Overlay Specifications 可以让内容逐步显示。
  \begin{columns}
    \column{0.5\textwidth}
    \begin{itemize}
      \item<1-> 第一项
      \item<2-> 第二项 
      \item<3-> 第三项
    \end{itemize}
    \only<2>{只和第二项一起显示。}
    \column{0.5\textwidth}
    \begin{itemize}
      \item<3-> \alert<3>{第一项}
      \item<4-> 第二项
      \item<5-> 第三项
    \end{itemize}
  \end{columns}
\end{frame}

\begin{frame}{Overlay Specifications}
  \framesubtitle<2-3>{我也可以被 Overlay 控制}
  许多样式也可以被 Overlay 控制。
  \begin{itemize}
    \item<1-> \textbf<1>{First item.}
    \item<2-> \textit<2>{Second item.}
    \item<3-> \textcolor<3>{blue}{Third item.}
    \item<4-> Fourth item.
  \end{itemize}
\end{frame}

\begin{frame}{表格上的动画}
  \begin{table}
    \begin{tabular}{l | c | c | c | c }
    Competitor Name & Swim & Cycle & Run & Total \\
    \hline \hline
    John T & 13:04 & 24:15 & 18:34 & 55:53 \onslide<2-> \\ 
    Norman P & 8:00 & 22:45 & 23:02 & 53:47 \onslide<3->\\
    Alex K & 14:00 & 28:00 & n/a & n/a \onslide<4->\\
    Sarah H & 9:22 & 21:10 & 24:03 & 54:35 
    \end{tabular}
    \caption{在表格上使用 onslide 创建动画。}
  \end{table}
\end{frame}

\subsection{Pause 命令}

\begin{frame}{Pause 命令}
  Pause 命令能够让内容逐步显示。
  \begin{itemize}
    \item First item. 
    \item Second item. \pause 在此停顿!\pause
    \item Third item.
  \end{itemize}
\end{frame}

\setbeamercovered{transparent}
\begin{frame}{只是透明}
  调用 \texttt{setbeamercovered} 命令可以让未显示的内容透明,而不是隐藏。
  \begin{itemize}
    \item<1-> First item.
    \item<2-> Second item.
    \item<3-> Third item.
    \item 未显示的内容是透明的。
  \end{itemize}
\end{frame}

\subsection{更自动的 Overlay Specifications}

\begin{frame}
  \frametitle{自动让 Item 逐步显示}
  \begin{itemize}[<+->]
    \item First item.
    \item Second item.
    \item Third item.
  \end{itemize}
\end{frame}

\begin{frame}
  \frametitle{自动让 Item 逐步显示}
  \framesubtitle{并且高亮显示}
  \begin{itemize}[<+-| alert@+>]
    \item First item.
    \item Second item.
    \item Third item.
  \end{itemize}
\end{frame}

\section{引用}

\begin{frame}{引用参考文献}
  真的有人会在幻灯片上引用参考文献吗?如果非要引用 \cite{Knuth92},使用带作者和年份的格式也许会更好。
\end{frame}

\begin{frame}[allowframebreaks]
  \frametitle{参考文献}
  % 删掉下一行,则只保留有 \cite 的参考文献
  \nocite{*}
  \bibliography{demo}
  \bibliographystyle{apalike}
\end{frame}

\end{document}